% ============================================================================
% Artículo: {{TITULO}}
% Autor:    {{AUTOR}}
% Fecha:    {{FECHA}}
% ============================================================================
\documentclass[12pt, a4paper]{article}

% --- Idioma y codificación ---
\usepackage[utf8]{inputenc}
\usepackage[T1]{fontenc}
\usepackage[spanish]{babel}

% --- Tipografía ---
\usepackage{lmodern}
\usepackage{microtype}

% --- Márgenes y geometría ---
\usepackage[
  top=2.5cm,
  bottom=2.5cm,
  left=3cm,
  right=3cm
]{geometry}

% --- Matemáticas ---
\usepackage{amsmath, amssymb, amsthm}

% --- Figuras y gráficos ---
\usepackage{graphicx}
\graphicspath{{figuras/}}
\usepackage{float}
\usepackage{caption}
\usepackage{subcaption}

% --- Tablas ---
\usepackage{booktabs}
\usepackage{array}

% --- Colores y enlaces ---
\usepackage[dvipsnames]{xcolor}
\usepackage[
  colorlinks=true,
  linkcolor=NavyBlue,
  citecolor=ForestGreen,
  urlcolor=RoyalBlue
]{hyperref}

% --- Código fuente (descomenta si lo necesitas) ---
% \usepackage{listings}
% \usepackage{minted}

% --- Bibliografía ---
\usepackage[
  backend=biber,
  style=apa,
  sorting=nyt
]{biblatex}
\addbibresource{referencias.bib}

% --- Encabezados y pies de página ---
\usepackage{fancyhdr}
\pagestyle{fancy}
\fancyhf{}
\fancyhead[L]{\small\leftmark}
\fancyhead[R]{\small\thepage}
\renewcommand{\headrulewidth}{0.4pt}

% --- Teoremas y definiciones ---
\theoremstyle{definition}
\newtheorem{definicion}{Definición}[section]
\newtheorem{ejemplo}{Ejemplo}[section]
\theoremstyle{plain}
\newtheorem{teorema}{Teorema}[section]
\newtheorem{proposicion}{Proposición}[section]

% ============================================================================
% Metadatos
% ============================================================================
\title{{{TITULO}}}
\author{{{AUTOR}}}
\date{{{FECHA}}}

% ============================================================================
\begin{document}

\maketitle

\begin{abstract}
  Resumen del artículo. Describe brevemente el objetivo, la metodología
  y las conclusiones principales.
\end{abstract}

\tableofcontents
\newpage

% ============================================================================
\section{Introducción}
% ============================================================================

Escribe aquí la introducción del artículo.

% ============================================================================
\section{Marco teórico}
% ============================================================================

Desarrollo del marco teórico y revisión de literatura.

% ============================================================================
\section{Metodología}
% ============================================================================

Descripción de la metodología utilizada.

% ============================================================================
\section{Resultados}
% ============================================================================

Presentación y análisis de los resultados.

% Ejemplo de figura:
% \begin{figure}[H]
%   \centering
%   \includegraphics[width=0.8\textwidth]{ejemplo.png}
%   \caption{Descripción de la figura.}
%   \label{fig:ejemplo}
% \end{figure}

% Ejemplo de tabla:
% \begin{table}[H]
%   \centering
%   \caption{Descripción de la tabla.}
%   \label{tab:ejemplo}
%   \begin{tabular}{lcc}
%     \toprule
%     Columna 1 & Columna 2 & Columna 3 \\
%     \midrule
%     Dato 1 & Dato 2 & Dato 3 \\
%     \bottomrule
%   \end{tabular}
% \end{table}

% ============================================================================
\section{Conclusiones}
% ============================================================================

Conclusiones del artículo.

% ============================================================================
% Bibliografía
% ============================================================================
\printbibliography

\end{document}
