% ============================================================================
% Presentación: {{TITULO}}
% Autor:        {{AUTOR}}
% Fecha:        {{FECHA}}
% ============================================================================
\documentclass[aspectratio=169, 12pt]{beamer}

% --- Idioma y codificación ---
\usepackage[utf8]{inputenc}
\usepackage[T1]{fontenc}
\usepackage[spanish]{babel}

% --- Tema y colores ---
% Temas disponibles: metropolis, Madrid, Berlin, CambridgeUS, etc.
% Descomenta el que prefieras:
\usetheme{metropolis}
% \usetheme{Madrid}
% \usetheme{Berlin}

% --- Personalización de colores ---
% Si usas metropolis:
\definecolor{PrimaryColor}{HTML}{2C3E50}
\definecolor{AccentColor}{HTML}{E74C3C}
\setbeamercolor{frametitle}{bg=PrimaryColor}
\setbeamercolor{progress bar}{fg=AccentColor}
\setbeamercolor{alerted text}{fg=AccentColor}

% --- Tipografía ---
\usepackage{lmodern}

% --- Matemáticas ---
\usepackage{amsmath, amssymb}

% --- Figuras y gráficos ---
\usepackage{graphicx}
\graphicspath{{figuras/}}
\usepackage{tikz}

% --- Tablas ---
\usepackage{booktabs}

% --- Bibliografía ---
\usepackage[
  backend=biber,
  style=phys,
  sorting=none,
  maxnames=2
]{biblatex}
\addbibresource{referencias.bib}

% --- Configuración adicional ---
\usepackage{appendixnumberbeamer}  % Reinicia numeración en apéndice

% Desactiva los botones de navegación
\setbeamertemplate{navigation symbols}{}

% ============================================================================
% Metadatos
% ============================================================================
\title{{{TITULO}}}
\subtitle{Subtítulo de la presentación}
\author{{{AUTOR}}}
\institute{Institución}
\date{{{FECHA}}}

% ============================================================================
\begin{document}

% --- Portada ---
\begin{frame}
  \titlepage
\end{frame}

% --- Índice ---
\begin{frame}{Contenido}
  \tableofcontents
\end{frame}

% ============================================================================
\section{Introducción}
% ============================================================================

\begin{frame}{Contexto}
  \begin{itemize}
    \item Primer punto de contexto.
    \item Segundo punto de contexto.
    \item Tercer punto de contexto.
  \end{itemize}
\end{frame}

\begin{frame}{Objetivo}
  \begin{alertblock}{Objetivo principal}
    Describe aquí el objetivo central de la presentación.
  \end{alertblock}
\end{frame}

% ============================================================================
\section{Desarrollo}
% ============================================================================

\begin{frame}{Punto clave 1}
  Contenido del primer punto.

  % Ejemplo de columnas:
  % \begin{columns}[T]
  %   \column{0.5\textwidth}
  %     Texto a la izquierda.
  %   \column{0.5\textwidth}
  %     \includegraphics[width=\textwidth]{ejemplo.png}
  % \end{columns}
\end{frame}

\begin{frame}{Punto clave 2}
  Contenido del segundo punto.

  % Ejemplo de tabla:
  % \begin{table}
  %   \centering
  %   \begin{tabular}{lcc}
  %     \toprule
  %     Item & Valor A & Valor B \\
  %     \midrule
  %     Uno  & 10      & 20 \\
  %     Dos  & 30      & 40 \\
  %     \bottomrule
  %   \end{tabular}
  % \end{table}
\end{frame}

\begin{frame}{Punto clave 3}
  Contenido del tercer punto.

  % Ejemplo de bloques:
  % \begin{exampleblock}{Ejemplo}
  %   Un ejemplo ilustrativo.
  % \end{exampleblock}
\end{frame}

% ============================================================================
\section{Resultados}
% ============================================================================

\begin{frame}{Resultados principales}
  \begin{enumerate}
    \item Primer resultado.
    \item Segundo resultado.
    \item Tercer resultado.
  \end{enumerate}
\end{frame}

% ============================================================================
\section{Conclusiones}
% ============================================================================

\begin{frame}{Conclusiones}
  \begin{itemize}
    \item Primera conclusión.
    \item Segunda conclusión.
    \item Trabajo futuro.
  \end{itemize}
\end{frame}

% --- Slide final ---
\begin{frame}[standout]
  ¡Gracias!

  \vspace{1cm}
  {\small ¿Preguntas?}
\end{frame}

% --- Referencias ---
\appendix
\begin{frame}[allowframebreaks]{Referencias}
  \printbibliography
\end{frame}

\end{document}
