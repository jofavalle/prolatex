% ============================================================================
% Ensayo: {{TITULO}}
% Autor:  {{AUTOR}}
% Fecha:  {{FECHA}}
% ============================================================================
\documentclass[12pt, a4paper]{report}

% --- Idioma y codificación ---
\usepackage[utf8]{inputenc}
\usepackage[T1]{fontenc}
\usepackage[spanish]{babel}

% --- Tipografía ---
\usepackage{lmodern}
\usepackage{microtype}
\usepackage{setspace}
\onehalfspacing

% --- Márgenes y geometría ---
\usepackage[
  top=2.5cm,
  bottom=2.5cm,
  left=3cm,
  right=3cm
]{geometry}

% --- Matemáticas ---
\usepackage{amsmath, amssymb}

% --- Figuras y gráficos ---
\usepackage{graphicx}
\graphicspath{{figuras/}}
\usepackage{float}
\usepackage{caption}

% --- Tablas ---
\usepackage{booktabs}
\usepackage{array}

% --- Colores y enlaces ---
\usepackage[dvipsnames]{xcolor}
\usepackage[
  colorlinks=true,
  linkcolor=NavyBlue,
  citecolor=ForestGreen,
  urlcolor=RoyalBlue
]{hyperref}

% --- Bibliografía ---
\usepackage[
  backend=biber,
  style=apa,
  sorting=nyt
]{biblatex}
\addbibresource{referencias.bib}

% --- Encabezados y pies de página ---
\usepackage{fancyhdr}
\pagestyle{fancy}
\fancyhf{}
\fancyhead[L]{\small\leftmark}
\fancyhead[R]{\small\thepage}
\renewcommand{\headrulewidth}{0.4pt}
\renewcommand{\chaptermark}[1]{\markboth{#1}{}}

% --- Epígrafes (citas al inicio de capítulos) ---
\usepackage{epigraph}
\setlength{\epigraphwidth}{0.6\textwidth}

% ============================================================================
% Metadatos
% ============================================================================
\title{
  \vspace{-2cm}
  {\LARGE {{TITULO}}} \\[0.5cm]
  {\large Ensayo}
}
\author{{{AUTOR}}}
\date{{{FECHA}}}

% ============================================================================
\begin{document}

\maketitle
\thispagestyle{empty}
\newpage

\tableofcontents
\newpage

% ============================================================================
\chapter{Introducción}
% ============================================================================

% \epigraph{Una cita relevante.}{--- Autor de la cita}

Escribe aquí la introducción del ensayo. Presenta el tema, la tesis
central y una breve descripción de la estructura del documento.

% ============================================================================
\chapter{Desarrollo}
% ============================================================================

\section{Primer argumento}

Desarrollo del primer argumento o idea principal.

\section{Segundo argumento}

Desarrollo del segundo argumento o idea principal.

\section{Discusión}

Contraste de ideas, análisis crítico y síntesis de los argumentos
presentados.

% ============================================================================
\chapter{Conclusiones}
% ============================================================================

Conclusiones del ensayo. Retoma la tesis central y resume los hallazgos
o reflexiones más importantes.

% ============================================================================
% Bibliografía
% ============================================================================
\printbibliography[heading=bibintoc, title={Referencias}]

% ============================================================================
% Apéndices (descomenta si los necesitas)
% ============================================================================
% \appendix
% \chapter{Material complementario}
% Contenido del apéndice.

\end{document}
